\documentclass[11pt, a4paper]{moderncv}
\usepackage[ngerman]{babel}   %Paket für deutsche Spracheinstellungen.
\usepackage[utf8]{inputenc}  %Das Dokument soll UTF-8-Format nutzen (wegen deutschen Umlauten).
\usepackage[left=2.4cm, right=1.0cm, top=1.7cm, bottom=2.0cm]{geometry}  %Seitenrand anpassen.
 %\usepackage{fontawesome}  %Paket für schönere Symbole bei den persönlichen Daten.
 
\moderncvtheme[blue]{classic}
 %\moderncvicons{awesome}
\moderncvicons{awesome}
 \newlength\mylen
 \addtolength\mylen{\linewidth}
 \addtolength\mylen{-\hintscolumnwidth}
 \addtolength\mylen{-\separatorcolumnwidth}
 \addtolength\mylen{-3mm}
%\definecolor{mycolor}{RGB}{255,51,76}
\definecolor{frontColor}{rgb}{0.22,0.45,0.70}% light blue
\definecolor{backColor}{RGB}{200,200,200}% grey
\usepackage{tikz}
\newcommand{\grade}[1]{%
	\begin{tikzpicture}
	\clip (1em-.4em,-.35em) rectangle (5em +.5em ,1em);
	\foreach \x in {1,2,...,5}{
		%\draw (\x em,0) circle (.35em);
		\path[{fill=backColor}] (\x em,0) circle (.35em);
	}
	\begin{scope}
	\clip (1em-.4em,-.35em) rectangle (#1em +.5em ,1em);
	\foreach \x in {1,2,...,5}{
		\path[{fill=frontColor}] (\x em,0) circle (.35em);
	}
	\end{scope}
	
	\end{tikzpicture}%
}
 %%%%%%%%%%%%%%%%%%%%%%%%%%%%%%%%%%%%%%%%%%%%%%%%%%%%%%%%%%%%%%%%%%%
 \newcommand\tab[1][0.1cm]{\hspace*{#1}}
%%%%%%%%%%%%%%%%%%%%%%%%%%%%%%%%%%%%%%%%%%%%%%%%%%%%%%%%%%%%%%%%%%%%%
% \renewcommand*{\listitemsymbol}{\hspace{2em}\labelitemi~}
% \newcommand\mybitem[1]{%
% 	\parbox[t]{3mm}{\textbullet}\parbox[t]{\mylen}{#1}}
 \newcommand\mybitem[1]{%
 	\parbox[t]{3mm}{\textbullet}\parbox[t]{10cm}{#1}\\[1.6mm]}
 
 %Die persönlichen Daten:
\name{Jürgen}{Buchner}
\title{Lebenslauf}
\address{Kaiserschützenstraße 8}{4690 Schwanenstadt}
\phone[mobile]{+43\,699\,17952841}
\email{juergenbuchner@gmail.com}
%\homepage{www.meine-homepage.de}
%\social[twitter][www.twitter.com]{Twitter-Name}
%\social[linkedin][www.linkedin.com]{LinkedIn-Name}
\extrainfo{\httplink[\faXingSquare~Juergen\_Buchner9]{xing.de/profile/Juergen_Buchner9}}
\social[github][www.github.com/JuergenBuchner]{JuergenBuchner}
%\extrainfo{Hier können extra Informaitonen stehen.}
\photo[3cm]{foto}  %Die Dateiendung wird weggelassen.
%\quote{"Hier kann mein Lebensmotto oder ein Zitat stehen."}
 
\begin{document}
  \makecvtitle
  
  \section{Persönliche Daten}
  \cvline{Geburtstag:}{20.10.1990}
  \cvline{Geburtsort:}{Gmunden}
 
  \section{Ausbildung}
  \cventry{10/2016 -- 09/2018}{Master Studium Automatisierungstechnik}{FH OÖ}{Wels}{}{Schwerpunkt: Industrielle Informatik (INIF)}
  \cventry{10/2013 -- 09/2016}{Bachelor Studium Automatisierungstechnik}{FH OÖ}{Wels}{}{Schwerpunkt: Industrielle Informatik (INIF)\\
  	Bachelorarbeit: ``Semi-Autonomous Identification and
  	Execution of Dexterity Tasks''
  	\\Bachelorarbeit 1: ``Sensor system for inspection and
  	exploration with low visibility in a disaster environment"
  }
  \cventry{10/2013}{Lehre mit Matura}{HTBLA}{Vöcklabruck}{}{Fachbereich: Maschinenbau}
  \cventry{09/2016 -- 05/2010}{Lehre zum Mechatroniker}{Inocon Technologie GmbH}{Attnang-Puchheim}{}{Tätigkeitsbereich: Sondermaschinenbau}
%  \cventry{1992--1993}{Gymnasium}{Schulname}{Schulort}{Leistungsfächer: Fach1 und Fach2}{Abitur: 1,0}
  
%  \section{Publikationen}
%  \cvlistitem{Meine 1. Publikation.}
%  \cvlistitem{Noch eine Publikation.}
  \section{Beruflicher Werdegang}
  \cventry{10/2017 -- 01/2018}{Praktikum: Masterprojekt 2}{TGW Logistics Group}{Wels}{}{Aufgabenstellung: Entwicklung einer Datenaufzeichnungssoftware für Condition Monitoring bei Regalbediengeräten}
  \cventry{03/2017 -- 06/2017}{Praktikum: Masterprojekt 1}{FH OÖ}{Wels}{}{Aufgabenstellung: Aufschwingen eines Doppelpendels 
  	(flachheitsbasierte Regelung und Zweipunkt-Grenzwertproblem eines nichtlinearen unteraktuierten 
  	Systems)
  }
  \cventry{02/2011 -- 09/2013}{Vollzeit: Monteur im Sondermaschinenbau}{Inocon Technologie GmbH}{Attnang-Puchheim}{}{Aufgaben: 
  	\begin{itemize}
  		\setlength{\itemindent}{.5cm}
  \item{Projektleitung mech. Montage von Sondermaschinen}
  \item{Fertigung von Schweißkonstruktionen }
  \item{Aufbau von mechanischen Komponenten } 
  \item{Zusammenbau von voll automatisierten Anlagen} 
  \item{Inbetriebnahme von Pneumatik Systemen} 
  \item{Montage und Service im In- und Ausland}
  	\end{itemize}
	}

  \section{Zivildienst}
  \cventry{05/2010 -- 01/2011}{Zivildienst}{Lebenshilfe OÖ - Tagesheimstätte}{Gmunden}{}{}
  \section{Technische Kenntnisse}
  \cvcomputer{\textsc{Matlab} \& Simulink}{\grade{4} \tab \textit{sehr gute Kenntnisse}}{Mathematica}{\grade{3} \tab \textit{gute Kenntnisse}}
  \cvcomputer{Microsoft Office}{\grade{3} \tab \textit{gute Kenntnisse}}{TeXstudio}{\grade{3} \tab \textit{gute Kenntnisse}}
  \cvcomputer{Visual Studio IDE}{\grade{3} \tab \textit{gute Kenntnisse}}{Spyder IDE}{\grade{3} \tab \textit{gute Kenntnisse}}
  \cvcomputer{GitHub}{\grade{2} \tab \textit{Grundkenntnisse}}{SVN}{\grade{1} \tab \textit{Grundkenntnisse}}
  \cvcomputer{NI Vision Assistant}{\grade{1} \tab \textit{Grundkenntnisse}}{ROS Framework}{\grade{1} \tab \textit{Grundkenntnisse}}
  \subsection{Programmiersprachen}
  \cvcomputer{C\#}{\grade{4} \tab \textit{sehr gute Kenntnisse}}{Java}{\grade{3} \tab \textit{gute Kenntnisse}}
  \cvcomputer{Python}{\grade{3} \tab \textit{gute Kenntnisse}}{Javascript}{\grade{3} \tab \textit{gute Kenntnisse}}
  \cvcomputer{Assembler, C}{\grade{1} \tab \textit{Grundkenntnisse}}{Sprachen nach IEC 61131-3}{\grade{2} \tab \textit{Grundkenntnisse}}
  
  \section{Sprachkenntnisse}
%  \cvlanguage{Deutsch}{Muttersprache}{}
  \cvlanguage{Deutsch}{\grade{5} \tab \textit{Muttersprache}}{}
  \cvlanguage{Englisch}{\grade{4}\tab \textit{fließend in Wort und Schrift}}{}{\begin{itemize}
  		\setlength{\itemindent}{4cm}
  		\item Schulunterricht und FH Lehrveranstaltungen
  		\item Auslandssemester in New York, USA
  	\end{itemize}
  }
  \section{Führungsqualitäten}
  \cventry{10/2013 -- 09/2018}{Jahrgangssprecher}{FH OÖ}{Wels}{}{
  	\begin{itemize}
  		\setlength{\itemindent}{.5cm}
  	\item Organisation von Prüfungsterminen und Terminverschiebungen
  	\item Teilnahme an LVA Evaluierungsgesprächen mit dem Dekan
  	\item Teilnahme an Seminaren zur Verbesserung des Automatisierungstechnik Curriculums mit Professoren und Interessenvertretern aus der Wirtschaft
	\end{itemize}
	}
  \section{Auszeichnungen}
  \cventry{06/2015}{Marshall Plan Scholarship}{Austrian Marshall Plan Foundation}{Wien}{}{}
  \cventry{02/2017}{INNOVATIONaward}{FH OÖ}{Wels}{}{Kategorie: „Bestes Fächerübergreifendes Studentenprojekt“}
  \section{Hobbys und Interessen}
  \cvlistdoubleitem{Wandern/Bergsteigen}{Klettern}
  \cvlistdoubleitem{Wakeboarden}{Snowboarden}
  \cvlistdoubleitem{Volleyball}{Tischtennis}
  
  \vfill  %Platz bis Seitenende auffüllen.
  Jürgen Buchner\\
  Schwanenstadt, \today
\end{document}